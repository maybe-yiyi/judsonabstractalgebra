\documentclass[11pt]{article}
\usepackage{fancyhdr}
\usepackage[margin=1in]{geometry}
\usepackage{enumerate}
\usepackage[shortlabels]{enumitem}

\usepackage{amsmath}
\usepackage{amssymb}
\usepackage{amsthm}

\pagestyle{fancy}
\fancyhead[l]{Yiyoung Liu}
\fancyhead[c]{Abstract Algebra: Theory and Applications}
\fancyhead[r]{Judson}
\renewcommand{\headrulewidth}{0.2pt}
\setlength{\headheight}{15pt}

\newtheorem{theorem}{Theorem}
\newtheorem{lemma}[theorem]{Lemma}

\newcommand{\exercise}[1]{\textbf{\textit{Exercise #1.}}}
\newcommand{\N}{\mathbb{N}}
\newcommand{\Z}{\mathbb{Z}}
\newcommand{\Q}{\mathbb{Q}}
\newcommand{\R}{\mathbb{R}}
\newcommand{\C}{\mathbb{C}}
\newcommand{\group}[2]{(#1, #2)}
\newcommand{\cyclicsg}[1]{\langle #1 \rangle}
\newcommand{\order}[1]{|#1|}

\begin{document}
\section*{Chapter 3}
\exercise{1}
\begin{enumerate}[(a)]
	\item False. $U(8)$ is not cyclic. We can verify this manually by noticing that the order of every element is 2, and so no individual element can generate the group $U(8)$ of order 4.
	\item False. The generators of $Z_{60}$ are all of the integers $1 \leq r < 60$ such that $\gcd(r, n) = 1$, and we can notice that $\gcd(49, 60) = 1$ but 49 is composite.
	\item False. Any element $a$ that could be a generator for $\Q$, the element $a/2$ is not generated by $a$.
	\item False. Consider $S_3$. Every subgroup is cyclic, but the group is not.
	\item True. Suppose $G$ is infinite. Consider an element $g \in G$. $\cyclicsg{g}$ cannot be finite, since that would imply there are infinitely many subgroups. But $\cyclicsg{g}$ cannot be infinite, since then $\cyclicsg{g^n}$ would provide infinite subgroups over all $n \in \Z^+$. Therefore if there are finitely many subgroups the group must be finite.
\end{enumerate}
\exercise{2}
\begin{enumerate}[(a)]
	\item $\order{5} = 12$
	\item $\order{\sqrt3} = \infty$
	\item $\order{\sqrt3} = \infty$
	\item $\order{-i} = 4$
	\item $\order{72} = 10$
	\item $\order{312} = 157$
\end{enumerate}
\exercise{3}
\begin{enumerate}[(a)]
	\item $\{\ldots, -14, -7, 0, 7, 14, \ldots\}$
	\item $\{0, 3, 6, 9, 12, 15, 18, 21\}$
	\item $\{0\}$, $\{0, 6\}$, $\{0, 4, 8\}$, $\{0, 3, 6, 9\}$, $\{0, 2, 4, 6, 8, 10\}$, $\Z_{12}$
	\item $\{0\}$, $\{0, 30\}$, $\{0, 20, 40\}$, $\{0, 15, 30, 45\}$, $\{0, 12, 24, 36, 48\}$, $\{0, 10, 20, 30, 40, 50\}$, $\{0, 6, 12, \ldots, 48, 54\}$, $\{0, 5, 10, \ldots, 50, 55\}$, $\{0, 4, 8, \ldots, 52, 56\}$, $\{0, 3, 6, \ldots, 54, 57\}$, $\{0, 2, 4, \ldots, 56, 58\}$, $\Z_{60}$
	\item $\{0\}$, $\Z_{13}$
	\item $\{0\}$, $\{0, 24\}$, $\{0, 16, 32\}$, $\{0, 12, 24, 36\}$, $\{0, 8, 16, 24, 32, 40\}$, $\{0, 6, 12, \ldots, 36, 42\}$, $\{0, 4, 8, \ldots, 40, 44\}$, $\{0, 3, 6, \ldots, 42, 45\}$, $\{0, 2, 4, \ldots, 44, 46\}$, $\Z_{48}$
	\item $\{1, 3, 7, 9\}$
	\item $\{1, 5, 7, 11, 13, 17\}$
	\item $\{\ldots, 1/49, 1/7, 1, 7, 49, \ldots\}$
	\item $\{1, i, -1, -i\}$
	\item $\{\ldots -1/4, -i/2, 1, 2i, -4\}$
	\item $\{1, e^{i\pi/4}, e^{i\pi/2}, \ldots, e^{3i\pi/2}, e^{7i\pi/8}\}$
	\item $\{1, e^{i\pi/3}, e^{2i\pi/3}, e^{i\pi}, e^{4i\pi/3}, e^{5i\pi/3}\}$
\end{enumerate}
\exercise{4}
\begin{enumerate}[(a)]
	\item $\left\{\begin{bmatrix} 1 & 0 \\ 0 & 1 \end{bmatrix}, \begin{bmatrix} 0 & 1 \\ -1 & 0 \end{bmatrix}, \begin{bmatrix} -1 & 0 \\ 0 & -1 \end{bmatrix}, \begin{bmatrix} 0 & -1 \\ 1 & 0 \end{bmatrix}\right\}$
	\item $\{\begin{bmatrix} 1 & 0 \\ 0 & 1 \end{bmatrix}, \begin{bmatrix} 0 & 1/3 \\ 3 & 0 \end{bmatrix}\}$
	\item $\left\{\begin{bmatrix} 1 & 0 \\ 0 & 1 \end{bmatrix}, \begin{bmatrix} 1 & -1 \\ 1 & 0 \end{bmatrix}, \begin{bmatrix} 0 & -1 \\ 1 & -1 \end{bmatrix}, \begin{bmatrix} -1 & 0 \\ 0 & -1 \end{bmatrix}, \begin{bmatrix} -1 & 1 \\ -1 & 0 \end{bmatrix}, \begin{bmatrix} 0 & 1 \\ -1 & 1 \end{bmatrix}\right\}$
	\item $\left\{\ldots, \begin{bmatrix} 1 & 1 \\ 0 & 1 \end{bmatrix}, \begin{bmatrix} 1 & 0 \\ 0 & 1 \end{bmatrix}, \begin{bmatrix} 1 & -1 \\ 0 & 1 \end{bmatrix}, \ldots, \begin{bmatrix} 1 & -n \\ 0 & 1 \end{bmatrix}\right\}$
	\item $\left\{\begin{bmatrix} 1 & 0 \\ 0 & 1 \end{bmatrix}, \begin{bmatrix} 1 & -1 \\ -1 & 0 \end{bmatrix}, \begin{bmatrix} 2 & -1 \\ -1 & 1 \end{bmatrix}, \begin{bmatrix} 3 & -2 \\ -2 & 1 \end{bmatrix}, \begin{bmatrix} 5 & -3 \\ -3 & 2 \end{bmatrix}, \ldots, \begin{bmatrix} \phi_{n+2} & -\phi_{n+1} \\ -\phi_{n+1} & \phi_{n} \end{bmatrix}, \ldots\right\}$ where $\phi_{n}$ is the $n$th Fibonacci number
	\item $\left\{\begin{bmatrix} 1 & 0 \\ 0 & 1 \end{bmatrix}, \begin{bmatrix} \sqrt3/2 & 1/2 \\ -1/2 & \sqrt3/2 \end{bmatrix}, \begin{bmatrix} 1/2 & \sqrt3/2 \\ -\sqrt3/2 & 1/2 \end{bmatrix}, \ldots, \begin{bmatrix} \sqrt3/2 & -1/2 \\ 1/2 & \sqrt3/2 \end{bmatrix}\right\}$
\end{enumerate}
\exercise{5} \\
0: 1;\quad 1: 18;\quad 2: 9;\quad 3: 6;\quad 4: 9;\quad 5: 18;\quad 6: 3;\quad 7: 18;\quad 8: 9;\quad 9: 2;\quad 10: 9;\quad 11: 18;\quad 12: 3;\quad 13: 18;\quad 14: 9;\quad 15: 6;\quad 16: 9;\quad 17: 18 \\\\
\exercise{6} \\
$id: 1;\quad \rho_1: 4;\quad \rho_2: 2;\quad \rho_3: 4;\quad \mu_1: 2;\quad \mu_2: 2;\quad \delta_1: 2;\quad \delta_2: 2$ \\\\
\exercise{7} \\
$\{1\}, \{1, -1\}, \{1, I, -1, -I\}, \{1, J, -1, -J\}, \{1, K, -1, -K\}$ \\\\
\exercise{8} \\
$\{1\}, \{1, 7, 13, 19\}, \{1, 11\}, \{1, 17, 19, 23\}, \{1, 19\}, \{1, 29\}$ \\\\
\exercise{9} \\
4, 12, 20, 28 \\\\
\exercise{10}
\begin{enumerate}[(a)]
	\item 0
	\item 1, -1
	\item 1, -1
\end{enumerate}
\exercise{11} \\
$\order{a} = 1, 2, 3, 4, 6, 8, 12, 24$ \\\\
\exercise{12} \\
Cyclic group with 1 generator: $\Z_2$ \\
Cyclic group with 2 generators: $\Z_3$ \\
Cyclic group with 4 generators: $\Z_5$ \\
It is impossible to find a cyclic group for any $n$ generators, since the number of generators of $\Z_m$ is given by $\phi(m)$, Euler's totient function. There are some $n$ for which $\phi(m) = n$ does not exist (the first being 14). \\\\
\exercise{13} \\
The only groups $U(n)$ for $n \leq 20$ that are not cyclic are $n = 8, 12, 15, 16, 20$.
\begin{proof} $U(n)$ will be cyclic if and only if $n = 1, 2, 4, p^k$, or $2p^k$, where $p$ is an odd prime number. \\
I dunno how to do this proof.
\end{proof}
\noindent\exercise{14} \\
See from exercise 4 that $A$ and $B$ have finite orders from parts (a) and (c) respectively. We can notice that $AB$ gives us the matrix in part (d), which did not have finite order. \\\\
\exercise{15}
\begin{enumerate}[(a)]
	\item $3i - 3$
	\item $8 - i$
	\item $43 - 18i$
	\item 82
	\item $i$
	\item 2
\end{enumerate}
\exercise{16}
\begin{enumerate}[(a)]
	\item $\sqrt{3} + i$
	\item $5/\sqrt2 + 5i/\sqrt2$
	\item -3
	\item $1/2\sqrt2 - i/2\sqrt2$
\end{enumerate}
\exercise{17}
\begin{enumerate}[(a)]
	\item $\sqrt2\operatorname{cis}(7\pi/4)$
	\item $5\operatorname{cis}(\pi)$
	\item $2\sqrt2\operatorname{cis}(\pi/4)$
	\item $2\operatorname{cis}(\pi/6)$
	\item $3\operatorname{cis}(3\pi/2)$
	\item $4\operatorname{cis}(\pi/6)$
\end{enumerate}
\exercise{18}
\begin{enumerate}[(a)]
	\item $\frac{1 - i}{2}$
	\item $8i$
	\item $-16\sqrt3 + 16i$
	\item -1
	\item $-\frac{1}{4}$
	\item -4096
	\item $128 - 128i$
\end{enumerate}
\exercise{19}
\begin{enumerate}[(a)]
	\item $\sqrt{a^2 + b^2} = \sqrt{a^2 + (-b)^2}$
	\item $(a + bi)(a - bi) = a^2 + b^2$
	\item $\frac{1}{a + bi} = \frac{a - bi}{(a + bi)(a - bi)}$
\end{enumerate}
\end{document}